\documentclass[a4paper,12pt]{article}
\usepackage[utf8]{inputenc}
\usepackage[T1]{fontenc}
\usepackage{booktabs}
\usepackage{natbib}
\usepackage{amsmath}
\usepackage[colorlinks=true]{hyperref}



\title{Testing the unguarded X hypothesis}
\begin{document}
\maketitle
\section{Introduction}
The unguarded X (UX) hypothesis states that one reason for females to live longer than males is that they are usually the homogametic sex. This implies that either somatic or inherited (partially) recessive deleterious mutations in the X chromosome are always exposed in males, but concealed in heterozygous females. This hypothesis requires recessive mutations, maintained by mutation-selection balance, that negatively affect longevity both in males and in females. This scenario is hardly controversial. The main uncertainty is whether this kind of mutations can fully explain the difference in longevity between sexes.

The alternative hypothesis is that a different kind of mutations is required to account for the difference in longevity between males and females. Namely, sexually antagonistic mutations, which are maintained at relatively high frequencies in the population by a selection balance. Sexually antagonistic mutations could happen in any chromosome. However, there are theoretical reasons and evidence to belive that they are especially frequent in the X chromosome \citep{Gibson2002}.

\citet{Kelly1999} suggested an experiment to determine if low-frequency, deleterious mutations can explain the observed genetic variance in a character that is correlated with fitness. The alternative is that high-frequency mutations maintained by a selection balance are required to explain the observed variance. The experiment is relatively easy and it involves a short selection experiment and a short inbreeding experiment. If most genetic variance in the character is due to a mutation-selection balance, the response to selection, small as it may be, will be comparable to the effect of inbreeding. However, in the case of any selection balance (antagonistic pleiotropy or overdominance), the response to selection is expected to be very low, while the effect of inbreeding relatively high. Needless to say that the conclusion would only apply to the population used in the experiment.

Kelly probably did not have in mind a sex-specific character when he suggested such an experiment. And he did not attempt to guess on what chromosome the relevant variants were located. Thus, we need to make sure if the UX hypothesis can be tested with a similar experiment, and exactly how it should be performed.

\bibliographystyle{abbrvnat}
\bibliography{ux}

\end{document}
